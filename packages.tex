% Pacotes usados neste modelo e suas respectivas configura��es

%=====================================================

% formato dos arquivos-fonte (usar UTF8 no Linux e ISO no Windows)
%\usepackage[latin1]{inputenc}	% se arquivos LaTeX estiverem em ISO-8859-1
\usepackage[isolatin]{inputenc} % se arquivos LaTeX estiverem em ISO-8859-1
%\usepackage[utf8]{inputenc}	% se arquivos LaTeX estiverem em Unicode

%=====================================================

% packages sugeridos para facilitar sua reda��o
\usepackage[brazil]{babel}	% linguagem do texto � portugu�s
\usepackage{graphicx}		% incluir figuras em PS e EPS
\usepackage{alltt,moreverb}	% mais comandos no modo verbatim
\usepackage[obeyspaces]{url}	% inclusao de URLs na bibliografia

%=====================================================

% pacote para formata��o de c�digo-fonte
\usepackage{listings}

\lstset{language=c}
\lstset{inputencoding=isolatin,extendedchars=true}
\lstset{basicstyle=\ttfamily\footnotesize,commentstyle=\textit,stringstyle=\ttfamily}
\lstset{showspaces=false,showtabs=false,showstringspaces=false}
\lstset{numbers=left,stepnumber=1,numberstyle=\tiny}
\lstset{columns=flexible,mathescape=true}
\lstset{frame=single}

%=====================================================

% pacote para formata��o de algoritmos
\usepackage{algorithm,algorithmic}

\floatname{algorithm}{Algoritmo}
\renewcommand{\algorithmiccomment}[1]{~~~// #1}
%\algsetup{linenosize=\footnotesize,linenodelimiter=.}

%=====================================================

